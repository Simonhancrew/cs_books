% can't be compiled successfully
% 勾股定理
\documentclass{article}
\usepackage{CJKutf8}
\usepackage[T1]{fontenc}
\begin{document}
\begin{CJK}{UTF8}{gkai}

\section{勾股定理在古代}
西方称公股定理为毕达哥拉斯定理, 将公股定理的发现归功于公前6世纪的毕达哥拉斯
学派. 该学派得到了一个法则, 可以求出可排成直角三角形三边的三元数组.
毕达哥拉斯学派没有书面著作, 该定理的严格表述和证明则见于
欧几里德<<几何原本>>的命题47: ``直角三角形斜边上的正方形等于两直角边上的两个正方形之和''.
\section{勾股定理的近代形式}
\end{CJK}
\end{document}
